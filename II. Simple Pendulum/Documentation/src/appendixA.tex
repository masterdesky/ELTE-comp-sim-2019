\section*{Appendix A}

\subsection*{A.1.\ \ A gerjesztett-csillapított kettős inga}

A gerjesztett-csillapított kettős inga mozgásegyenletei felírhatóak a következő\cite{double} módon:

\begin{equation}
    \ddot{\theta_{1}}
    =
    {
    \frac{m_{2} L_{1} \omega_{1}^{2} * \sin{\left( 2 * d \theta \right)}
    +
    2 m_{2} L_{2} \omega_{1}^{2} * \sin{\left( d \theta \right)}
    +
    2 g m_{2} * \cos{\left( \theta_{2} \right)} * \sin{\left( d \theta \right)}
    +
    2 g m_{1} * \sin{\left( \theta \right)}
    +
    \gamma_{1}
    }{
    -2 L_{1} * \left( m_1 + m_2 * \sin^{2}{\left( d \theta \right)} \right)
    }
    }
\end{equation}

\begin{equation}
    \ddot{\theta_{2}}
    =
    {
    \frac{m_{2} L_{2} \omega_{2}^{2} * \sin{\left( 2 * d \theta \right)}
    +
    2 * \left( m_{1} + m_{2} \right) * L_{2} \omega_{1}^{2} * \sin{\left( d \theta \right)}
    +
    2 g * \left( m_{1} + m_{2} \right) * \cos{\left( \theta_{1} \right)} * \sin{\left( d \theta \right)}
    +
    \gamma_{2}
    }{
    -2 L_{2} * \left( m_1 + m_2 * \sin^{2}{\left( d \theta \right)} \right)
    }
    }
\end{equation}
Ahol $m$, $L$, $\theta$ és $\omega$, megfelelő indexekkel ellátott mennyiségek rendre az adott inga tömegét, kötélhosszát, kitérésének szögét, valamint szögsebességét jelölik. Az $1$-es index közvetlenül a felfüggesztéshez, míg a $2$-es az alsó, magához az elsőhöz rögzített ingát jelöli. A $g$ mennyiség a gravitációs gyorsulás. \\
Többek között még a mozgásegyenletekben szereplő $\gamma_{1}$ és $\gamma_{2}$ mennyiségek az alábbiakat takarják:

\begin{equation}
    \gamma_{1}
    =
    2 \alpha - 2 \beta * \cos{\left( d \theta \right)}
\end{equation}
\begin{equation}
    \gamma_{2}
    =
    2 \alpha * \cos{\left( d \theta \right)}
    -
    \frac{2 * \left( m_{1} + m_{2} \right)}{m_{2}} \beta
\end{equation}
Ahol használtuk az első testre ható
\begin{equation}
    \alpha
    =
    q_{1} * \omega_{1}
    -
    F_{D_{1}} * \sin{\left( \Omega_{D_{1}} * t \right)}
\end{equation}
valamint a második testre ható
\begin{equation}
    \beta
    =
    q_{2} * \omega_{2}
    -
    F_{D_{2}} * \sin{\left( \Omega_{D_{2}} * t \right)}
\end{equation}
gerjesztő-csillapító hatások jelölését. Ezekben $\Omega_{D}$ és $F_{D}$ - a megfelelő indexekkel - a gerjesztés frekvenciáját és amplitúdóját jelöli, míg az itt és fentebb megjelenő $d \theta = \theta_{1} - \theta_{2}$, a két inga kitérésének különbsége.