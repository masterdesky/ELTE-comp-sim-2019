\section*{Appendix A - Ábrák} \label{A}
\topskip0pt
\vspace*{\fill}
\begin{center}
    \includegraphics[width=\textwidth]{images/logistic_model.pdf}
    \captionof{figure}{A logisztikus modell differenciálegyenletének megoldását adó egyik lehetséges görbesereg $w_{be} \in \left[ 0.5, 1 \right]$ születési rátára, $w_{ki} = 0.3$ halálozási ráta, $n_0 = 10$ kezdeti egyedszám és $k = 100$ maximális egyedszám mellett.} \label{fig:1}
\end{center}
\begin{center}
    \includegraphics[width=\textwidth]{images/connected_logistic_model.pdf}
    \captionof{figure}{A csatolt-logisztikus modell differenciálegyenletének megoldását adó egyik lehetséges görbesereg $w_{be_{1}} \in \left[ 0.5, 1 \right]$-re, ahol $w_{be_{1}}$ a sorszám szerinti első faj születési rátája, $w_{be_{2}} = 0.6$ születési, $w_{ki_{1}} = w_{ki_{2}} = 0.3$ halálozási ráták, $n_{0_{1}} = 10$, $n_{0_{2}} = 20$ kezdeti egyedszámok és $k_{1} = k_{2} = 100$ maximális egyedszám, valamint $\alpha = 0.1$ és $\beta = 0.9$ kölcsönhatási tényezők mellett.} \label{fig:2}
\end{center}
\vspace*{\fill}
\newpage
\topskip0pt
\vspace*{\fill}
\begin{center}
    \includegraphics[width=\textwidth]{images/lv_model.pdf}
    \captionof{figure}{A Lotka--Volterra-modell differenciálegyenletének megoldását adó egyik lehetséges görbe, $n_{0_{r}} = 400$, $n_{0_{f}} = 200$ kezdeti egyedszámok, valamint $a = 0.4$, $b = c = 0.004$, $d = 0.9$ fejlődési ráták mellett, $k \to \infty$, $s = 0$ határesetben.} \label{fig:3}
\end{center}
\begin{center}
    \includegraphics[width=\textwidth]{images/lv_model_more.pdf}
    \captionof{figure}{A Lotka--Volterra-modell differenciálegyenletének megoldását adó egyik lehetséges görbesereg, $a \in \left[ 0.1, 1.1 \right]$ paraméter esetén, $n_{0_{r}} = 400$, $n_{0_{f}} = 200$ kezdeti egyedszámok, valamint $b = c = 0.004$, $d = 0.9$ fejlődési ráták mellett, $k \to \infty$, $s = 0$ határesetben.} \label{fig:4}
\end{center}
\vspace*{\fill}
\newpage
\topskip0pt
\vspace*{\fill}
\begin{center}
    \includegraphics[width=\textwidth]{images/connected_logistic_model_stability.pdf}
    \captionof{figure}{A csatolt-logisztikus modell stabilitásának vizsgálata, $k_{1} \in \left[ 50, 150 \right]$-re, $k_{2} = 100$ maximális egyedszám, $w_{be_{1}} = w_{be_{2}} = 0.3$ születési és $w_{ki_{1}} = w_{ki_{2}} = 0.3$ halálozási ráták, $n_{0_{1}} = n_{0_{2}} = 10$ kezdeti egyedszám, valamint $\alpha = \beta = 1$ kölcsönhatási tényezők mellett. A vastag fekete sáv a $k_{1} = k_{2}$ esetnek megfelelő görbét jelöli, mely esetében a rendszer instabil egyensúlyi helyzetben tartózkodik.} \label{fig:5}
\end{center}
\vspace*{\fill}
\newpage
\topskip0pt
\vspace*{\fill}
\begin{center}
    \includegraphics[width=0.83\textwidth]{images/connected_logistic_model_animals.pdf}
    \captionof{figure}{A csatolt-logisztikus modell populációinak változása, $n_{1} - n_{2}$ diagramon ábrázolva, $w_{be_{1}} \in \left[ 0.5, 1 \right]$-re, ahol $w_{be_{1}}$ a sorszám szerinti első faj születési rátája, $w_{be_{2}} = 0.6$ születési, $w_{ki_{1}} = w_{ki_{2}} = 0.3$ halálozási ráták, $n_{0_{1}} = 10$, $n_{0_{2}} = 20$ kezdeti egyedszámok és $k_{1} = k_{2} = 100$ maximális egyedszám, valamint $\alpha = 0.1$ és $\beta = 0.9$ kölcsönhatási tényezők mellett.} \label{fig:6}
\end{center}
\vspace*{\fill}
\newpage
\topskip0pt
\vspace*{\fill}
\begin{center}
    \includegraphics[width=0.83\textwidth]{images/connected_logistic_model_stability_animals.pdf}
    \captionof{figure}{A csatolt-logisztikus modell populációinak változása, $n_{1} - n_{2}$ diagramon ábrázolva, $k_{1} \in \left[ 50, 150 \right]$-re, $k_{2} = 100$ maximális egyedszám, $w_{be_{1}} = w_{be_{2}} = 0.3$ születési és $w_{ki_{1}} = w_{ki_{2}} = 0.3$ halálozási ráták, $n_{0_{1}} = 10$, $n_{0_{2}} = 20$ kezdeti egyedszámok, valamint $\alpha = \beta = 1$ kölcsönhatási tényezők mellett.} \label{fig:7}
\end{center}
\vspace*{\fill}
\newpage
\topskip0pt
\vspace*{\fill}
\begin{center}
    \includegraphics[width=\textwidth]{images/lv_model_animals.pdf}
    \captionof{figure}{A Lotka--Volterra-modell populációinak változása, $n_{r} - n_{f}$ diagramon ábrázolva, $a \in \left[ 0.1, 1.1 \right]$ paraméter esetén, $n_{0_{r}} = 400$, $n_{0_{f}} = 200$ kezdeti egyedszámok, valamint $b = c = 0.004$, $d = 0.9$ fejlődési ráták mellett, $k \to \infty$, $s = 0$ határesetben.} \label{fig:8}
\end{center}
\vspace*{\fill}
\newpage
\topskip0pt
\vspace*{\fill}
\begin{center}
    \includegraphics[width=\textwidth]{images/lv_model_drag_animals.pdf}
    \captionof{figure}{A Lotka--Volterra-modell populációinak változása, $n_{r} - n_{f}$ diagramon ábrázolva, $a \in \left[ 0.1, 1.1 \right]$ paraméter esetén, $n_{0_{r}} = 300$, $n_{0_{f}} = 200$ kezdeti egyedszámok, valamint $b = c = 0.004$, $d = 0.9$ fejlődési ráták mellett, $k = 800$, $s = 0$ esetben.} \label{fig:9}
\end{center}
\vspace*{\fill}