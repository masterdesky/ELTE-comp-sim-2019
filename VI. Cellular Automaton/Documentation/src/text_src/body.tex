\selectlanguage{english}
\begin{abstract}
    \noindent A \emph{Számítógépes szimulációk} laboratórium hatodik és egyben utolsó gyakorlatának alkalmával a sejtautomaták kérdéskörébe tettünk bepillantást. Feladatunk volt, hogy létrehozzuk az elhíresült Conway-féle Életjáték megvalósító kódot, mely tetszőleges kihalási szabály szerinti működést képes produkálni. Ezután létre kellett hoznunk egy homokdombautomatát és meg kellett határoznunk annak skálázási törvényében szereplő exponens értékét.
\end{abstract}
\selectlanguage{magyar}

\begin{multicols}{2}

\section{Bevezető} \label{sec:1}
A labor utolsó gyakorlatán a sejtautomaták témakörével foglalkoztunk. Ezeket populárisan híressé a Conway-féle Életjáték tette, azonban már évtizedekkel előtte is ismertek voltak a számításelmélet területén, melynek alapjait Neumann János fektette le. \\
A sejtautomaták mechanizmusának szakirodalma tág, azonban a kurzus során ebben nem kellett a témába mélyebben belemennünk, így kizárólagosan a Conway-féle Életjáték alapvető szabályaival, valamint az Abel-, vagy más néven Bak--Tang--Wiesenfeld-féle homokdomb modell megvalósításával foglalkoztunk.

\section{Feladatok és elmélet} \label{sec:2}
A mostani gyakorlat alkalmával - az előzőhöz hasonlóan - szintén magunknak kellett megírjunk minden forráskódot, előre csak a feladatokat ismertető leírás volt megadva. A két megvalósítandó rendszer a fentebb is említett Életjáték nevű sejtautomata, valamint a homokdomb modell volt.

\subsection{Életjáték} \label{sub:2.1}
Az Életjátékot 1970-ben John Horton Conway, brit matematikus találta fel. A játék alapvető felépítését egy klasszikus, 2-dimenziós, végtelen négyzetrácson működő sejtautomata alkotta, mely működésére specifikus szabályok vonatkoztak. Az játékterület négyzetrácsán minden négyzet két állapotot vehet fel, melyeket a $0$ és $1$ értékekkel, vagy a \texttt{Halott} és \texttt{Élő} kifejezésekkel különböztetünk meg. Egy játék kezdetén a négyzterács minden cellájának kijelölünk valamilyen értéket a fenti kettő közül, a játék pedig ebből az általunk választott tetszőleges kezdőpozícióból indul. Ezt követően a négyzetrács az adott szabály szerint lépésekben frissül. A Conway-féle Életjáték standard módon megfogalmazott szabályai a következőek:

\begin{enumerate}
    \item Ha egy cella \texttt{Élő}, és \texttt{Élő} szomszédjainak száma $0$, vagy $1$, akkor a következő körben \texttt{Halott}á változik. (Alulnépesedés)
    \item Ha egy cella \texttt{Élő}, és \texttt{Élő} szomszédjainak száma $2$, vagy $3$, akkor a következő körben is \texttt{Élő} marad. (Fajfenntartás)
    \item Ha egy cella \texttt{Halott}, és \texttt{Élő} szomszádjainak száma pontosan $3$, akkor a következő körben \texttt{Élő}vé változik. (Reprodukció)
    \item Ha egy cella \texttt{Élő}, és \texttt{Élő} szomszédjainak száma több, mint $3$, akkor a következő körben is \texttt{Halott}á változik. (Túlnépesedés)
    \item Minden más esetben a cella \texttt{Halott}.
\end{enumerate}
A program megírása során ezeket az értékeket kontrollálhatóvá kellett megírjuk. Ez azt takarta, hogy tetszőleges, terminálból megadható $N$ érték esetén történjen a játékban reprodukció a fenti $3$ helyett, valamint $N-1$ és $N$ értéknél pedig maradjon a cella \texttt{Élő}, ha eddig is \texttt{Élő} volt. A másik két szabály hasonlóan ehhez kellett ilyenkor igazodjon automatikusan. \\
A másik megkötés az volt, hogy a programot tetszőlegesen, különböző kezdőfeltételekkel indíthassuk (végtelen, periodikus, konstans élő határ és konstans random határ), hasonlóan terminálból szabályotható módon.

\subsection{Abel-féle 2D homokdomb modell} \label{sub:2.2}
A 2 dimenziós homokdomb modell a nevéből is érthetően \q{homokszem-tornyok} 2 dimenziós síkon vett ábrázolása. A síkot felosztva egy négyzetráccsal, minden négyzetet megfeleltetünk 1-1 homokszem-toronynak, melyben $N$ darab homokszem tartózkodik. A rendszert egy $H$ mátrixxal írhatjuk ilyenkor le, melynek minden eleme az adott négyzetben tartózkodó homokszemek számát jelöli. \\
Az Abel-féle definícióban egy ilyen tornyot instabilnak hívunk, ha benne több, mint $3$ homokszem található. Ha egy torony eléri ezt az instabil állapotot, akkor egy lavina alakul ki, mely a következőt jelenti:

\begin{equation}
    \text{Ha} \ M_{i,j} > 3
    \to
    \begin{cases}
        M_{i,j} = M_{i,j} - 4 \\
        M_{i\pm1,j} = M_{i\pm1,j} + 1 \\
        M_{i,j\pm1} = M_{i,j\pm1} + 1 \\
    \end{cases}
\end{equation}
A gyakorlat során ezen modell megvalósítása volt a feladatunk. Ezután ellenőriznünk kellett annak skálázási törvényét. Megfigyelés alapján az $N \left( n_{t} \right)$ eloszlást az alábbi függvény adja meg:

\begin{equation}
    N \left( n_{t} \right)
    =
    \frac{1}{n_{t}^{b}}
\end{equation}

\section{Megvalósítás} \label{sec:3}
Az alapvető forráskódokat a kötelező előírásnak megfelelően \texttt{C++} nyelven implementáltam. Mind a sejtautomata, mind pedig a homokdombmodell egy-egy külön forrásfileban található. A forráskódokat az eddigiekhez hasonlóan egy saját batch file segítségével, benne a \texttt{clang} fordító felhasználásával fordítottam. A futtatható \texttt{exe} programokat egy Jupyter Notebook-ban futó Python 3 kernel segítségével indítottam, a szimuláció a kezdőfeltételeket szintén ebből a környezetből várja, bemenő paraméterek formájában.
\\ \\
A kimenet minden esetben egy \texttt{.dat} file, melyben a szimulációt lépéseit illusztráló 2D mátrixok egyszerűen egymás alatt szerepelnek. A gyakorlaton tárgyalt feladatokhoz néhány animációt is készítettem, amik közül azonban a homokdomb esetén nem az eredeti forráskódot, hanem egy könnyebben meghívható és léptethető python-verziót használtam. Az animációk elkészítéséhez így két különböző kódot használtam. A sejtautomatát animáló kód egy, az eddigiekhez is hasonló, az \texttt{imageio} könyvtárat és az \texttt{ffmpeg} szoftver használtam. A homokdomb esetében egy teljesen más megközelítést alkalmaztam. Egy \texttt{class SandpileAnimation} nevű python classt írva és a \texttt{matplotlib.animation} library-t felhasználva egy tetszőlegesen léptethető és videót generáló kódot valósítottam meg, mely képes arra, hogy pontosan a homokdomb egyensúlyi pozíciójának beálltáig animáljon, így optimalizálva egy szimulációt.
\\ \\
A sejtautomata programnak kétféle bemeneti módja létezik. Az első esetében egy tetszőleges méretű és pozíciójú, random generált életet helyezhet el a játékos a játéktéren. A második verzió esetén egy \texttt{.dat} file-ból adhat be neki egy $0$ és $1$ értékekből álló, tetszőleges méretű mátrixot, melyet aztán szintén általa választott pozicióban helyezhet el a játéktéren. A kiértékelés közben kiderült, hogy az MS Paint program megfelelő ilyen tetszőleges kezdeti állapotok létrehozására, hisz benne tetszőleges módon rajzolhatunk fekete és fehér pixelekből felépített képeket, melyeket aztán $0$ és $1$ értékekként értelmezve, szintén beadhatunk a programnak.
\\ \\
A homokdomb modell esetén két fajta kezdőfeltételt lehet váalsztani, azonban a peremfeltételeket minden esetben nyíltként értelmeztem. Az első kezdőfeltételt választva a homokszemek száma az egyes tornyokban véletlenszerűen választódik. A homokszemek száma a $\left[ 0, n \right]$ intervallumból sorsolódik ahol $n$ egy terminálból beadható paraméter. A második kezdőfeltétel választása esetén minden toronyban egyaránt $n$ darab homokszem fog elhelyezkedni kezdetben, a szimuláció ebből az állapotból indul. Ezen utóbbi a homokdomb modell szimulációk szokásos kezdőfeltétele, ahol $n$ értékét $7$-nek szokták megválasztani.
\\ \\
A végleges forráskódok és a programokat futtató Notebook file mind elérhető GitHub-on \citep{github}.

\section{Kiértékelés} \label{sec:4}
\subsection{Sejtautomata}
A sejtautomaták viselkedését legjobban működésének folyamatában vizsgálhatjuk. Emiatt több különböző esetéről animációt is készítettem, melyeket a YouTube-on elérhetővé is tettem, és melynek linkje a hivatkozások között elérhető\nocite{yt}. Az animációkon számtalan ismert automatát szimuláltam, különböző határfeltételek mellett. Ezek között sorrendet nem figyelve szerepel pl. a \emph{Gosper glider gun}, a \emph{Copperhead} űrhajó, vagy pl. a \emph{Kox's galaxy} névre hallgató oszcillátor. Ezen felsorolt hármat a jegyzőkönyvben is ábrázoltam és a (\ref{fig:1}) - (\ref{fig:3}) ábrákon tekinthetőek meg. \\
Abban a két esetben, amikor a határokat valamilyen konstans \text{Élő}, vagy random módon sorsolt elemeknek választjuk meg, a határokról az azonnal ki fog terjedni, ha a (\ref{sub:2.1})-es részben szereplő $N$ paraméter értéke $N \leq 3$. Minden más esetben nem változhatnak a határok ebből fakadóan, hisz a határon található sejtek melleti cellák mindegyike pontosan $3$ határnégyzettel szomszédos. \\
Komplexebb formákat magasabb rendű esetekben azonba nagyon nehéz kialakítani. $N = 8$ esetben pl. az egyetlen stabil forma az egész teret egyenletesen kitöltő \texttt{Élő} cellák halmaza, ez is kizárólag konstans élő, vagy periodikus határfeltételek mellett. Minden más esetben a sejtccsoport elhal. \\
Az $N \in \left[ 4, 7 \right]$ intervallumban már lehetséges létrehozni olyan formákat, amik stabilak. Ilyen pl. a $N = 5$ esetben periodikus határfeltételek mellett sakktáblaszerűen kitöltött játéktér, megfelelő oldalhosszakkal megválasztva.

\subsection{Homokdomb modell}
A megvalósított homokdomb modellt különböző nagyságú játékterületre futtatva, mértem a stabil állapot eléréséhez szükséges lépések számát. Minden futtatás esetén a kezdőfeltételt úgy választottam meg, hogy minden toronyban azonosan $n = 7$ homokszem tartózkodjon kezdetben. A szimulációt ezután a stabil állapot beálltáig futtattam, a kapott végleges állapotokat pedig a (\ref{fig:4}) - (\ref{fig:7}) képeken ábrázoltam, különböző játéktér méretekre. A mintázat minden esetben nagyon hasonló és tökéletesen visszaadja a \citep{csabaisandpile} leírásban is látható képet. Ehhez hasonlóan próbálkoztam olyan modellek elkészítésével is, melyek nem négyzet, hanem tetszőleges téglalap alaőúak voltak. Ezek külsőre nagyon hasonlóak voltak a négyzet alapúakhoz, így külön nem részletezem őket, hisz nem produkáltak semmiféle egyéb érdekességet. \\
A $120 \times 120$ méretű játéktérrel rendelkező esetről egy animációt is készítettem, melyet a YouTube-on elérhetővé is tettem, és melynek linkje a hivatkozások között elérhető\nocite{yt}.

\section{Diszkusszió} \label{sec:5}
A laborgyakorlat előtt már volt tapasztalatom a sejtautomaták működését és programozását illetően, azonban most még több ismeretre tehettem szert a gyakorlat feladatainak megoldása közben. Emellett megismerkedhettem a homokdomb modellel is és újabb módszereket próbálhattam ki annak animálásától motiválva. Amennyire időm engedte a szakdolgozat írása és a vizsgákra való készülés mellett, megpróbáltam kellő mélységben belemenni az utolsó gyakorlat feladatainak megvalósításába is. Több-kevesebb sikerrel, egyáltalán nem $100\%$-osan minden feladatot kidolgozva, de végül időre végeztem elégséges módon vele.

\end{multicols}