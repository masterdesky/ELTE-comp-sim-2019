\selectlanguage{english}
\begin{abstract}
    \noindent A \emph{Számítógépes szimulációk} laboratórium második alkalmával az egyszerű ingák (matematikai, csillapított, gerjesztett, fizikai) differenciálegyenleteinek numerikus megoldásait vizsgáltuk. Összehasonlítottuk az egyenleteket megoldó negyedrendű Runge--Kutta, a lépéshossz-váltó (adaptív) negyedrendű Runge--Kutta, a Runge--Kutta--Cash--Karp és az adatív Runge--Kutta--Cash--Karp, valamint az Euler és az Euler--Cromer módszereket. Kiegészítésként megvizsgáltuk ezek pontosságát és érzékenységét a kettős ingára vonatkozóan is. \\
\end{abstract}
\selectlanguage{magyar}

\begin{multicols}{2}

\end{multicols}